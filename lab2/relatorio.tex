\documentclass[a4paper,12pt]{article}

\usepackage[top=3cm, bottom=2cm, left=3cm, right=2cm]{geometry}
\usepackage[utf8]{inputenc}
\usepackage[portuguese]{babel}
\usepackage{booktabs}
\usepackage{multirow}
\usepackage{graphicx}
\usepackage{longtable}
\usepackage{verbatim}

\usepackage{float}
\floatstyle{ruled}
\newfloat{program}{thp}{lop}
\floatname{program}{Script}

\title{Serviços de Rede e de Sistema \\
Interior Routing }

\author{André Fernandes (ei03107) \and Miguel Gomes (ei07075) \and Pedro Batista (ext10392)}

\begin{document}

\maketitle

\section{Topologia}

\section{Cenários}

\section{Análise de Tráfego}

	Com o objectivo de capturar o tráfego completo da rede configurámos as capacidades Switched Port Analyzer (SPAN) do switch. Esta configuração permitiu que uma porta do switch servisse para receber dados de monitorização de todo o tráfego da rede, independentemente da VLAN em que os pacotes surjam.


	\begin{figure}[htp]
   	\begin{center}
	 		\includegraphics[height=300pt]{}
		\end{center}
		\caption{}
		\label{fig:}
	\end{figure}

\section{Arquivos de log}

\section{Referências}

SPAN: http://www.cisco.com/en/US/products/hw/switches/ps708/products_tech_note09186a008015c612.shtml#terms

\end{document}
