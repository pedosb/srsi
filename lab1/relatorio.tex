\documentclass[a4paper,12pt]{article}

\usepackage[top=3cm, bottom=2cm, left=3cm, right=2cm]{geometry}
\usepackage[utf8]{inputenc}
\usepackage[portuguese]{babel}
\usepackage{booktabs}
\usepackage{multirow}

\title{Serviços de Rede e de Sistema \\
Endereçamento IP e serviço DNS}

\author{André Fernandes (ei03107) \and Pedro Batista (ext10392)}

\begin{document}

\maketitle

\section{Endereçamento}
\subsection{Projectar o esquema de endereçamento da rede da empresa,
usando do bloco de endereços acima, apenas os blocos de endereços com
o tamanho necessário para cada rede. Justifique.}

\begin{table}[h]
   \centering
   \begin{tabular}{ l | c | c | c }
      \toprule
      \textbf{Descrição} & \textbf{Quantidade} & \textbf{Total} & \textbf{Intervalo de Endereços} \\\hline

      \multicolumn{4}{c}{\textit{Rede de Servidores (VLAN000)}} \\\hline
      Servidores & 3 & 3 & \ldots \\\hline

      \multicolumn{4}{c}{\textit{Rede Edifício Sede (VLAN100)}} \\ \hline
      Estações & 300 & 300 & 192.168.0.0 - 192.168.1.255 \\ \hline

      \multicolumn{4}{c}{\textit{Loja 1 }} \\\hline
      VLAN200 & 24 & \multirow{3}{*}{41} & \multirow{3}{*}{192.168.2.0 - 192.168.2.63} \\\cline{1-2}
      VLAN201 & 16 & \\\cline{1-2}
      Servidor & 1 & \\ \hline
      \multicolumn{4}{c}{\textit{Loja 2}} \\\hline
      VLAN210 & 24 & \multirow{3}{*}{41} & \multirow{3}{*}{192.168.2.64 - 192.168.2.127} \\\cline{1-2}
      VLAN211 & 16 & \\\cline{1-2}
      Servidor & 1 & \\ \hline
      \multicolumn{4}{c}{\textit{Loja 3}} \\\hline
      VLAN220 & 24 & \multirow{3}{*}{41} & \multirow{3}{*}{192.168.2.128 - 192.168.2.191} \\\cline{1-2}
      VLAN221 & 16 & \\\cline{1-2}
      Servidor & 1 & \\ \hline

      \multicolumn{4}{c}{\textit{Armazém (VLAN300)}} \\\hline
      Estações & 17 & \multirow{2}{*}{18} & \multirow{2}{*}{192.168.2.192 - 192.168.2.223} \\\cline{1-2}
      Servidor & 1 & \\ \bottomrule

   \end{tabular}
   \caption{Esquema de endereçamento.}
   \label{tab:enderecamento_esquema}
\end{table}

\subsection{Apresente os vários endereços (identificação da rede e broadcast) e as
respectivas máscaras para cada uma das redes.}
\subsection{Atribua endereços aos servidores indicados e às gateways de cada rede local.
As boas práticas recomendam que os servidores tenham os endereços mais
baixos da rede, por exemplo o mais baixo será o servidor de DNS, e a gateway
o mais alto (antes do broadcast).}

\begin{table}[ht]
   \centering
   \begin{tabular}{ l | c }
      \toprule
%      \textbf{Descrição} & \textbf{Rede} & \textbf{Broadcast} & \textbf{Mascara} \\\hline
      \multicolumn{2}{c}{Rede Edifício Sede} \\\hline 
      Rede & 192.168.0.0 \\\hline 
      Broadcast & 192.168.1.255 \\\hline 
      Mascara & 255.255.254.0 \\\hline
      Gateway & 192.168.1.254 \\\hline

      \multicolumn{2}{c}{Loja 1} \\\hline 
      Rede & 192.168.2.0 \\\hline 
      Broadcast & 192.168.2.63 \\\hline 
      Mascara & 255.255.255.192 \\\hline
      Gateway & 192.168.2.62 \\\hline
      Servidor (DNS e Proxy) & 192.168.2.1 \\\hline
      \multicolumn{2}{c}{Loja 2} \\\hline 
      Rede & 192.168.2.64 \\\hline 
      Broadcast & 192.168.2.127 \\\hline 
      Mascara & 255.255.255.192 \\\hline
      Gateway & 192.168.2.126 \\\hline
      Servidor (DNS e Proxy) & 192.168.2.65 \\\hline
      \multicolumn{2}{c}{Loja 3} \\\hline 
      Rede & 192.168.2.128 \\\hline 
      Broadcast & 192.168.2.191 \\\hline 
      Mascara & 255.255.255.192 \\\hline
      Gateway & 192.168.2.190 \\\hline
      Servidor (DNS e Proxy) & 192.168.2.129 \\\hline

      \multicolumn{2}{c}{Armazém} \\\hline 
      Rede & 192.168.2.192 \\\hline 
      Broadcast & 192.168.2.223 \\\hline 
      Mascara & 255.255.255.224 \\\hline
      Gateway & 192.168.2.222 \\\hline
      Servidor (Cache DNS) & 192.168.2.193 \\\hline
%      Loja 1 &             192.168.2.0   & 192.168.2.63  &      \multirow{3}{*}{255.255.255.192} \\\cline{1-3}
%      Loja 2 &             192.168.2.64  & 192.168.2.127 &      \\\cline{1-3}
%      Loja 3 &             192.168.2.128 & 192.168.2.191 &      \\\hline
%      Armazém &            192.168.2.192   & 192.168.2.223 &      255.255.254.224 \\
      \bottomrule
   \end{tabular}
   \caption{Subredes.}
   \label{tab:subredes}
\end{table}

\end{document}
