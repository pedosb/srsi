\documentclass[a4paper,12pt]{article}

\usepackage[top=3cm, bottom=2cm, left=3cm, right=2cm]{geometry}
\usepackage[utf8]{inputenc}
\usepackage[portuguese]{babel}
\usepackage{booktabs}
\usepackage{multirow}

\title{Serviços de Rede e de Sistema \\
	Endereçamento IP e serviço DNS}

	\author{André Fernandes (ei03107) \and Pedro Batista (ext10392)}

	\begin{document}

	\maketitle

	\section{Endereçamento}
	\subsection{Projectar o esquema de endereçamento da rede da empresa,
		usando do bloco de endereços acima, apenas os blocos de endereços com
			o tamanho necessário para cada rede. Justifique.}

			\begin{table}[h]
			\centering
			\begin{tabular}{ l | c | c | c }
			\toprule
			\textbf{Descrição} & \textbf{Quantidade} & \textbf{Total} & \textbf{Intervalo de Endereços} \\\hline
			
			\multicolumn{4}{c}{\textit{Rede Edifício Sede (VLAN100)}} \\ \hline
			Estações & 300 & 300 & 192.168.0.0 - 192.168.1.255 \\ \hline
				
			\multicolumn{4}{c}{\textit{Loja 1 (VLAN21X)}} \\\hline
			VLAN210 & 25 & 25 & 192.168.2.0 - 192.168.2.31 \\\hline
			VLAN211 & 17 & 17 & 192.168.2.32 - 192.168.2.63 \\\hline
				
			\multicolumn{4}{c}{\textit{Loja 2 (VLAN22X)}} \\\hline
			VLAN220 & 25 & 25 & 192.168.2.64 - 192.168.2.95 \\\hline
			VLAN221 & 17 & 17 & 192.168.2.96 - 192.168.2.127\\\hline
				
			\multicolumn{4}{c}{\textit{Loja 3 (VLAN23X)}} \\\hline
			VLAN230 & 25 & 25 & 192.168.2.128 - 192.168.2.159 \\\hline
			VLAN231 & 17 & 17 & 192.168.2.160 - 192.168.2.191 \\\hline
				
			\multicolumn{4}{c}{\textit{Armazém (VLAN300)}} \\\hline
			Estações & 17 & \multirow{2}{*}{18} & \multirow{2}{*}{192.168.2.192 - 192.168.2.223} \\\cline{1-2}
			Servidor & 1 & \\\hline 
			
			\multicolumn{4}{c}{\textit{DMZ (VLAN400)}} \\\hline
			Servidores & 3 & 3 & 20.49.51.160 - 20.49.51.168 \\\hline
			
			\multicolumn{4}{c}{\textit{Rede de Servidores (VLAN500)}} \\\hline
			Servidores & 3 & 3 & 192.168.2.224 - 192.168.2.231 \\\hline
			\bottomrule
			
			\end{tabular}
			\caption{Esquema de endereçamento.}
			\label{tab:enderecamento_esquema}
			\end{table}

			\subsection{Apresente os vários endereços (identificação da rede e broadcast) e as respectivas máscaras para cada uma das redes.}
				\subsection{Atribua endereços aos servidores indicados e às gateways de cada rede local.
					As boas práticas recomendam que os servidores tenham os endereços mais
						baixos da rede, por exemplo o mais baixo será o servidor de DNS, e a gateway
						o mais alto (antes do broadcast).}

						\begin{table}[ht]
						\centering
						\begin{tabular}{ l | c }
						\toprule
						%      \textbf{Descrição} & \textbf{Rede} & \textbf{Broadcast} & \textbf{Mascara} \\\hline
						\multicolumn{2}{c}{Rede Edifício Sede} \\\hline 
						Mascara & 255.255.254.0 \\\hline
						Rede & 192.168.0.0 \\\hline 
						Gateway & 192.168.1.254 \\\hline
						Broadcast & 192.168.1.255 \\\hline 

						\multicolumn{2}{c}{Loja 1} \\\hline 
						Mascara & 255.255.255.228 \\\hline
						
						SubRede1 & 192.168.2.0 \\\hline 
						Broadcast SubRede1 & 192.168.2.31 \\\hline
						Gateway SubRede1 & 192.168.2.30 \\\hline
					
						SubRede2 & 192.168.2.32 \\\hline 
						Broadcast SubRede2 & 192.168.2.63 \\\hline
						Gateway SubRede2 & 192.168.2.62 \\\hline
						
						Servidor SubRede1 (DNS e Proxy) & 192.168.2.1 \\\hline
						Servidor SubRede2 (DNS e Proxy) & 192.168.2.33 \\\hline
						
						\multicolumn{2}{c}{Loja 2} \\\hline 
						Mascara & 255.255.255.228 \\\hline
						
						SubRede1 & 192.168.2.68 \\\hline 
						Broadcast SubRede1 & 192.168.2.95 \\\hline
						Gateway SubRede1 & 192.168.2.94 \\\hline
					
						SubRede2 & 192.168.2.96\\\hline 
						Broadcast SubRede2 & 192.168.2.127 \\\hline
						Gateway SubRede2 & 192.168.2.126\\\hline
						
						Servidor SubRede1 (DNS e Proxy) & 192.168.2.69 \\\hline
						Servidor SubRede2 (DNS e Proxy) & 192.168.2.97 \\\hline
						
						\multicolumn{2}{c}{Loja 3} \\\hline 
						Mascara & 255.255.255.228 \\\hline
						
						SubRede1 & 192.168.2.128 \\\hline 
						Broadcast SubRede1 & 192.168.2.159 \\\hline
						Gateway SubRede1 & 192.168.2.154\\\hline
					
						SubRede2 & 192.168.2.160\\\hline 
						Broadcast SubRede2 & 192.168.2.191 \\\hline
						Gateway SubRede2 & 192.168.2.190 \\\hline
						
						Servidor SubRede1 (DNS e Proxy) & 192.168.2.129 \\\hline
						Servidor SubRede2 (DNS e Proxy) & 192.168.2.161 \\\hline
						
						\multicolumn{2}{c}{Armazém} \\\hline 
						Mascara & 255.255.255.224 \\\hline
						Rede & 192.168.2.192 \\\hline 
						Broadcast & 192.168.2.223 \\\hline 
						Gateway & 192.168.2.222 \\\hline
						Servidor (Cache DNS) & 192.168.2.193 \\\hline
						%      Loja 1 &             192.168.2.0   & 192.168.2.63  &      \multirow{3}{*}{255.255.255.192} \\\cline{1-3}
						%      Loja 2 &             192.168.2.64  & 192.168.2.127 &      \\\cline{1-3}
						%      Loja 3 &             192.168.2.128 & 192.168.2.191 &      \\\hline
						%      Armazém &            192.168.2.192   & 192.168.2.223 &      255.255.254.224 \\
						
						\multicolumn{2}{c}{Rede de Servidores Internos} \\\hline 
						Mascara & 255.255.255.248 \\\hline
						Rede & 192.168.2.224 \\\hline 
						Broadcast & 192.168.2.231 \\\hline 
						Gateway & 192.168.2.230 \\\hline
						Servidor DNS & 192.168.2.225 \\\hline
						Servidor WWW & 192.168.2.226 \\\hline
						Servidor MAIL & 192.168.2.227 \\\hline
						
						\multicolumn{2}{c}{DMZ} \\\hline 
						Mascara & 255.255.255.248 \\\hline
						Rede & 29.49.51.160 \\\hline 
						Broadcast & 20.49.51.167 \\\hline 
						Gateway & 20.49.51.166 \\\hline
						Servidor DNS & 20.49.51.161 \\\hline
						Servidor WWW & 20.49.51.162 \\\hline
						Servidor MAIL & 20.49.51.163 \\\hline
						\bottomrule
						\end{tabular}
						\caption{Subredes.}
						\label{tab:subredes}
						\end{table}

	\end{document}
